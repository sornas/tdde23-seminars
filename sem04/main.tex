\documentclass{beamer}

% vim: shiftwidth=2

\usecolortheme{seagull}
\usefonttheme{structurebold}

\usepackage[utf8]{inputenc}
\usepackage[swedish]{babel}

\usepackage{roboto}
\usepackage[scaled=0.9]{roboto-mono}
\usepackage[T1]{fontenc}

\usepackage{booktabs}
\usepackage{color}
\usepackage{icomma} % smart ',' in math mode
\usepackage{soul} % highlight and underline (etc)
\usepackage[normalem]{ulem} % strikethrought
\usepackage{url}

\usepackage{listings} % code listings

\usepackage{tikz}
\usetikzlibrary{tikzmark}

\setul{0.5ex}{0.3ex}

\title{Seminarium 04}
\subtitle{Experimentering}
\date{28 september 2021}
\author{Gustav Sörnäs}

\setlength{\parskip}{1em}
\renewcommand{\baselinestretch}{1.2}

\AtBeginSection[] % Do nothing for \section*
{
\begin{frame}<beamer>
\frametitle{Planering}
\tableofcontents[currentsection]
\end{frame}
}

\begin{document}
  \frame{\titlepage}

  \begin{frame}{Seminarieformen}

    Ibland läsa kod och diskutera frågor, ibland skriva kod i mindre grupper.
    Sedan diskussion i helklass.

    Innan seminariet: läs förberedelsematerialet och försök er på uppgifterna.

    Skicka in lösningar så vi kan diskutera i helklass:
    \texttt{seminarium.sörnäs.se}. Anonymt, sålänge du inte skriver ditt namn i
    koden :>

  \end{frame}

  \begin{frame}{Dagens seminarium}

    \begin{itemize}
      \item Uppgift: listbyggare
      \item Uppgift: git
      \item Uppgift: hjälp
      \item Uppgift: programutvecklingsprocessen
    \end{itemize}

  \end{frame}

  % fix for soul in beamer
% https://tex.stackexchange.com/questions/41683/why-is-it-that-coloring-in-soul-in-beamer-is-not-visible
\makeatletter
\let\HL\hl
\renewcommand\hl{%
  \let\set@color\beamerorig@set@color
  \let\reset@color\beamerorig@reset@color
  \HL}
\makeatother

\makeatletter
\let\UL\ul
\renewcommand\ul{%
  \let\set@color\beamerorig@set@color
  \let\reset@color\beamerorig@reset@color
  \UL}
\makeatother

\lstset{basicstyle=\ttfamily,language=python,escapeinside=||}

\begin{frame}[fragile]{Listbyggare}

  Övning 404: Konstruera en funktion \texttt{distribute} som tar ett element och
  en lista av listor, och lägger in elementet i varje dellista.

  \pause{}

  \begin{lstlisting}
def distribute(elem, lists):
    distributed = []
    for lst in lists:
        distributed.append(lst + [elem])
    return distributed
  \end{lstlisting}

  \pause{}

  \begin{lstlisting}
def distribute(elem, lists):
    return [lst + [elem] for lst in lists]
  \end{lstlisting}

\end{frame}

\begin{frame}[fragile]{Listbyggare, men med färger}

  Övning 404: Konstruera en funktion \texttt{distribute} som tar ett element och
  en lista av listor, och lägger in elementet i varje dellista.

  \begin{lstlisting}
def distribute(elem, lists):
    |\sethlcolor{orange}\hl{distributed = []}|
    for |\sethlcolor{cyan}\hl{lst in lists}|:
        |\sethlcolor{orange}\hl{distributed.append}|(|\sethlcolor{cyan}\hl{lst}| + [elem])
    return distributed
  \end{lstlisting}

  \begin{lstlisting}
def distribute(elem, lists):
    return [lst + [elem] for lst in lists]
  \end{lstlisting}

\end{frame}

\begin{frame}[fragile]{Listbyggare, men med färger}

  Övning 404: Konstruera en funktion \texttt{distribute} som tar ett element och
  en lista av listor, och lägger in elementet i varje dellista.

  \begin{lstlisting}
def distribute(elem, lists):
    distributed = []
    |\sethlcolor{cyan}\hl{for lst in lists}|:
        distributed.append(|\sethlcolor{orange}\hl{lst + [elem]}|)
    return distributed
  \end{lstlisting}

  \begin{lstlisting}
def distribute(elem, lists):
    return |\sethlcolor{orange}\hl{[lst + [elem]}| |\sethlcolor{cyan}\hl{for lst in lists}|]
  \end{lstlisting}

\end{frame}


  \begin{frame}{Listbyggare --- uppgift}

    \footnotesize

    \pause{}

    \begin{enumerate}[<+->]
      \item Alla tal större än eller lika med noll. \\
            \texttt{[1, 4, -2, 3.3] => [1, 4, 3.3]}

      \item Näst minsta talet i varje lista. Input är en lista av listor (på 1
            nivå, så ingen rekursion). Anta att det alltid finns minst två element. \\
            \texttt{[[1, 2, 3], [4, 6, 5, 7]] => [2, 5]} \\
            Tips: kolla på \texttt{sorted}.

      \item Alla ord som innehåller minst ett \texttt{a}, men alla \texttt{a}:n
            är utbytta mot \texttt{*}. \\
            \texttt{[''apa'', ''citron'', ''apelsin''] => [''*p*'', ''*pelsin'']}

      \item Alla tal 0--100 som är jämnt delbara med 3 eller 5, men
            \emph{inte} båda. \\
            \texttt{=> [3, 5, 6, 9, 10, 12, 18, 20, ...]}

      \item En 5x5-matris fylld med nollor. \\
            \texttt{=> [[0, 0, 0, 0, 0], [0, 0, 0, 0, 0], ... (x5 totalt)]}

      \item En 5x5-identitetsmatris, alltså en 5x5-matris med ettor på
            \emph{diagonalen} och nollor överallt annars. (Gör sig bättre uppritad.) \\
            \texttt{=> [[1, 0, 0, 0, 0], [0, 1, 0, 0, 0], ...]}

      % todo: gör i ordning några listiga lösningar

    \end{enumerate}

  \end{frame}

  \begin{frame}{git}

    git är i grunden ett verktyg för att spara olika versioner av hur filer såg
    ut vid olika tillfällen.

    Hittils: add - commit - push.

    \pause{}

    \begin{itemize}[<+->]
      \item add: markera filer som ''dom här vill jag spara''.
      \item commit: spara dom markerade filerna i en \emph{commit}.
      \item push: synkronisera mina commits med Gitlabs.
    \end{itemize}

  \end{frame}

  \begin{frame}{git - varför ni använder det}

    \pause{}

    \begin{itemize}[<+->]
      \item Det blir tröttsamt att maila kod till sig själv.
      \item Arbeta parallellt på egen dator / LiU:s datorer.
      \item Arbeta två+ personer parallellt. (TDDE25)
    \end{itemize}

  \end{frame}

  \begin{frame}{git - mer saker ni kan göra}

    \begin{itemize}
      \item pull: hämta commits från en server.
      \item log: se en lista över ändringar som skett.
      \item restore: återställ koden till en tidigare version.
    \end{itemize}

  \end{frame}

  \begin{frame}{git - hur ni tar er vidare}

    \pause{}

    Den officiella git-boken: \url{https://git-scm.com/book/en/v2}. Från vanlig
    användning till hur det fungerar på insidan.

    \pause{}

    En git-föreläsning jag var med och höll i vintras:
    \url{https://youtu.be/Db1XV8UTM1M}. Andra halvan inte lika relevant (än).
    
    \pause{}

    Labbassistenten!

  \end{frame}

  \begin{frame}{git - sökmotorer}

    Svar i forum-liknande format (som StackOverflow). Svarar ofta på en väldigt
    specifik fråga, men ni håller på att lära er dom absoluta grunderna.
    Avrådes om ni inte vet vad ni håller på med.

  \end{frame}

  \begin{frame}{Hjälp}

    Vad gör ni om ni fastnar med något i kursen?

    I vilken ordning?

    \pause{}

    Vad har ni tillgång till på tentan?

  \end{frame}

  \begin{frame}{Dokumentationen}

    \url{docs.python.org}

  \end{frame}

  %TODO
  \begin{frame}{Programutvecklingsprocessen}

  \end{frame}

\end{document}
