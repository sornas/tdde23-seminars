\documentclass{beamer}

\usecolortheme{seagull}
\usefonttheme{structurebold}

\usepackage[utf8]{inputenc}
\usepackage[swedish]{babel}

\usepackage{roboto}
\usepackage[scaled=0.9]{roboto-mono}
\usepackage[T1]{fontenc}

\usepackage{booktabs}
\usepackage{icomma} % smart ',' in math mode
\usepackage{minted} % code with syntax highlighting
\usepackage[normalem]{ulem} % strikethrought
\usepackage{url}

\title{Seminarium 04}
\subtitle{Experimentering}
\date{28 september 2021}
\author{Gustav Sörnäs}

\setlength{\parskip}{1em}
\renewcommand{\baselinestretch}{1.2}

\AtBeginSection[] % Do nothing for \section*
{
\begin{frame}<beamer>
\frametitle{Planering}
\tableofcontents[currentsection]
\end{frame}
}

\begin{document}
  \frame{\titlepage}

  \begin{frame}{Seminarieformen}

    Ibland läsa kod och diskutera frågor, ibland skriva kod i mindre grupper.
    Sedan diskussion i helklass.

    Innan seminariet: läs förberedelsematerialet och försök er på uppgifterna.

    Skicka in lösningar så vi kan diskutera i helklass:
    \texttt{seminarium.sörnäs.se}. Anonymt, sålänge du inte skriver ditt namn i
    koden :>

  \end{frame}

  \begin{frame}{Dagens seminarium}

    \begin{itemize}
      \item Uppgift: listbyggare
      \item Uppgift: git
      \item Uppgift: att få hjälp
      \item Uppgift: programutvecklingsprocessen
    \end{itemize}

  \end{frame}

  \begin{frame}{Listbyggare}

    \footnotesize

    \pause{}

    \begin{enumerate}[<+->]
      \item Alla tal större än eller lika med noll. \\
            \texttt{[1, 4, -2, 3.3] => [1, 4, 3.3]}

      \item Näst minsta talet i varje lista. Input är en lista av listor (på 1
            nivå, så ingen rekursion). Anta att det alltid finns minst två element. \\
            \texttt{[[1, 2, 3], [4, 6, 5, 7]] => [2, 5]} \\
            Tips: kolla på \texttt{sorted}.

      \item Alla ord som innehåller minst ett \texttt{a}, men alla \texttt{a}:n
            är utbytta mot \texttt{*}. \\
            \texttt{[''apa'', ''citron'', ''apelsin''] => [''*p*'', ''*pelsin'']}

      \item Alla tal 0--100 som är jämnt delbara med 3 eller 5, men
            \emph{inte} båda. \\
            \texttt{=> [3, 5, 6, 9, 10, 12, 18, 20, ...]}

      \item En 5x5-matris fylld med nollor. \\
            \texttt{=> [[0, 0, 0, 0, 0], [0, 0, 0, 0, 0], ... (x5 totalt)]}

      \item En 5x5-identitetsmatris, alltså en 5x5-matris med ettor på
            \emph{diagonalen} och nollor överallt annars. (Gör sig bättre uppritad.) \\
            \texttt{=> [[1, 0, 0, 0, 0], [0, 1, 0, 0, 0], ...]}

      % todo: gör i ordning några listiga lösningar

    \end{enumerate}

  \end{frame}

  \begin{frame}{git}

  \end{frame}

  \begin{frame}{Hjälp}

  \end{frame}

  \begin{frame}{Dokumentationen}

  \end{frame}

  \begin{frame}{Programutvecklingsprocessen}

  \end{frame}

\end{document}
