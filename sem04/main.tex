\documentclass{beamer}

% vim: shiftwidth=2

\usecolortheme{seagull}
\usefonttheme{structurebold}

\usepackage[utf8]{inputenc}
\usepackage[swedish]{babel}

\usepackage{roboto}
\usepackage[scaled=0.9]{roboto-mono}
\usepackage[T1]{fontenc}

\usepackage{booktabs}
\usepackage{color}
\usepackage{icomma} % smart ',' in math mode
\usepackage{soul} % highlight and underline (etc)
\usepackage[normalem]{ulem} % strikethrought
\usepackage{url}

\usepackage{listings} % code listings

\usepackage{tikz}
\usetikzlibrary{tikzmark}

\setul{0.5ex}{0.3ex}

\title{Seminarium 04}
\subtitle{Experimentering}
\date{28 september 2021}
\author{Gustav Sörnäs}

\setlength{\parskip}{1em}
\renewcommand{\baselinestretch}{1.2}

\AtBeginSection[] % Do nothing for \section*
{
\begin{frame}<beamer>
\frametitle{Planering}
\tableofcontents[currentsection]
\end{frame}
}

\begin{document}
  \frame{\titlepage}

  \begin{frame}{Seminarieformen}

    Ibland läsa kod och diskutera frågor, ibland skriva kod i mindre grupper.
    Sedan diskussion i helklass.

    Innan seminariet: läs förberedelsematerialet och försök er på uppgifterna.

    Skicka in lösningar så vi kan diskutera i helklass:
    \texttt{seminarium.sörnäs.se}. Anonymt, sålänge du inte skriver ditt namn i
    koden :>

  \end{frame}

  \begin{frame}{Dagens seminarium}

    \begin{itemize}
      \item Uppgift: listbyggare
      \item git
      \item Hjälp
      \item Dokumentation
      \item Uppgift: programutvecklingsprocessen
    \end{itemize}

  \end{frame}

  % fix for soul in beamer
% https://tex.stackexchange.com/questions/41683/why-is-it-that-coloring-in-soul-in-beamer-is-not-visible
\makeatletter
\let\HL\hl
\renewcommand\hl{%
  \let\set@color\beamerorig@set@color
  \let\reset@color\beamerorig@reset@color
  \HL}
\makeatother

\makeatletter
\let\UL\ul
\renewcommand\ul{%
  \let\set@color\beamerorig@set@color
  \let\reset@color\beamerorig@reset@color
  \UL}
\makeatother

\lstset{basicstyle=\ttfamily,language=python,escapeinside=||}

\begin{frame}[fragile]{Listbyggare}

  Övning 404: Konstruera en funktion \texttt{distribute} som tar ett element och
  en lista av listor, och lägger in elementet i varje dellista.

  \pause{}

  \begin{lstlisting}
def distribute(elem, lists):
    distributed = []
    for lst in lists:
        distributed.append(lst + [elem])
    return distributed
  \end{lstlisting}

  \pause{}

  \begin{lstlisting}
def distribute(elem, lists):
    return [lst + [elem] for lst in lists]
  \end{lstlisting}

\end{frame}

\begin{frame}[fragile]{Listbyggare, men med färger}

  Övning 404: Konstruera en funktion \texttt{distribute} som tar ett element och
  en lista av listor, och lägger in elementet i varje dellista.

  \begin{lstlisting}
def distribute(elem, lists):
    |\sethlcolor{orange}\hl{distributed = []}|
    for |\sethlcolor{cyan}\hl{lst in lists}|:
        |\sethlcolor{orange}\hl{distributed.append}|(|\sethlcolor{cyan}\hl{lst}| + [elem])
    return distributed
  \end{lstlisting}

  \begin{lstlisting}
def distribute(elem, lists):
    return [lst + [elem] for lst in lists]
  \end{lstlisting}

\end{frame}

\begin{frame}[fragile]{Listbyggare, men med färger}

  Övning 404: Konstruera en funktion \texttt{distribute} som tar ett element och
  en lista av listor, och lägger in elementet i varje dellista.

  \begin{lstlisting}
def distribute(elem, lists):
    distributed = []
    |\sethlcolor{cyan}\hl{for lst in lists}|:
        distributed.append(|\sethlcolor{orange}\hl{lst + [elem]}|)
    return distributed
  \end{lstlisting}

  \begin{lstlisting}
def distribute(elem, lists):
    return |\sethlcolor{orange}\hl{[lst + [elem]}| |\sethlcolor{cyan}\hl{for lst in lists}|]
  \end{lstlisting}

\end{frame}


  \begin{frame}{Listbyggare --- uppgift}

    \footnotesize

    \pause{}

    \begin{enumerate}[<+->]
      \item Alla tal större än eller lika med noll. \\
            \texttt{[1, 4, -2, 3.3] => [1, 4, 3.3]}

      \item Näst minsta talet i varje lista. Input är en lista av listor (på 1
            nivå, så ingen rekursion). Anta att det alltid finns minst två element. \\
            \texttt{[[1, 2, 3], [4, 6, 5, 7]] => [2, 5]} \\
            Tips: kolla på \texttt{sorted}.

      \item Alla ord som innehåller minst ett \texttt{a}, men alla \texttt{a}:n
            är utbytta mot \texttt{*}. \\
            \texttt{[''apa'', ''citron'', ''apelsin''] => [''*p*'', ''*pelsin'']}

      \item Alla tal 0--100 som är jämnt delbara med 3 eller 5, men
            \emph{inte} båda. \\
            \texttt{=> [3, 5, 6, 9, 10, 12, 18, 20, ...]}

      \item En 5x5-matris fylld med nollor. \\
            \texttt{=> [[0, 0, 0, 0, 0], [0, 0, 0, 0, 0], ... (x5 totalt)]}

      \item En 5x5-identitetsmatris, alltså en 5x5-matris med ettor på
            \emph{diagonalen} och nollor överallt annars. (Gör sig bättre uppritad.) \\
            \texttt{=> [[1, 0, 0, 0, 0], [0, 1, 0, 0, 0], ...]}

      % todo: gör i ordning några listiga lösningar

    \end{enumerate}

  \end{frame}

  \begin{frame}{git}

    git är i grunden ett verktyg för att spara olika versioner av hur filer såg
    ut vid olika tillfällen.

    Hittils: add - commit - push.

    \pause{}

    \begin{itemize}[<+->]
      \item add: markera filer som ''dom här vill jag spara''.
      \item commit: spara dom markerade filerna i en \emph{commit}.
      \item push: synkronisera mina commits med Gitlabs.
    \end{itemize}

  \end{frame}

  \begin{frame}{git - varför ni använder det}

    \pause{}

    \begin{itemize}[<+->]
      \item Det blir tröttsamt att maila kod till sig själv.
      \item Arbeta parallellt på egen dator / LiU:s datorer.
      \item Arbeta två+ personer parallellt. (TDDE25)
    \end{itemize}

  \end{frame}

  \begin{frame}{git - några vanliga saker ni kan göra}

    Hur gör ni dom här sakerna med git? Ta reda på valfritt sätt!

    \begin{itemize}
      \item Visa en lista av alla commits.
      \item Kolla vilka ändringar som skedde för 3 commits sedan.
      \item Återställ en fil med hur den såg ut i en tidigare commit.
      \item Kasta bort ändringar sedan senaste commiten.
      \item Skapa en commit med ändringar från bara vissa filer, inte alla.
    \end{itemize}
  \end{frame}

  \begin{frame}{git - hur ni tar er vidare}

    Den officiella git-boken: \url{https://git-scm.com/book/en/v2}. Från vanlig
    användning till hur det fungerar på insidan.

    \pause{}

    En git-föreläsning jag var med och höll i vintras:
    \url{https://youtu.be/Db1XV8UTM1M}. Andra halvan inte lika relevant (än).

    \pause{}

    Labbassistenten!

  \end{frame}

  \begin{frame}{Hjälp}

    Vad gör ni om ni fastnar med något i kursen?

    I vilken ordning?

    \pause{}

    Vad har ni tillgång till på tentan?

  \end{frame}

  \begin{frame}{Dokumentationen}

    \url{https://docs.python.org}

    \url{https://docs.opencv.org/2.4/modules/refman.html}

  \end{frame}

  \begin{frame}{Programutvecklingsprocessen}

    \begin{enumerate}
      \item Analys
      \item Specifikation
      \item Design
      \item Implementation
      \item Testning
      \item Underhåll
    \end{enumerate}

    \pause{}

    \begin{tabular}{r l}
      1--3: & Ni \\
      4--5: & Jag \\
      6: & Någon annan
    \end{tabular}

  \end{frame}

  \begin{frame}{Programutvecklingsprocessen --- uppgiften}

    Tentauppgift från förra året.

    \pause{}

    Skriv en funktion som exekverar ett program i det påhittade språket PyASM.

    Ett program består av en lista av instruktioner. En instruktion består av en
    tuple som innehåller en sträng och 1 eller fler argument.

    Programmet kan under körningen använda ett antal \emph{register} som är
    bokstäverna A--Z. Register innehåller heltal. Oanvända register har värdet
    0.

  \end{frame}

  \begin{frame}{Programutvecklingsprocessen --- instruktionerna}

    \begin{itemize}
      \item \texttt{LOG <register>} skriv ut värdet i registret.
      \item \texttt{SET <register> <value>} spara ett värde i ett register.
      \item \texttt{CPY <register> <source>} kopiera värdet från ett register till ett annat.
      \item \texttt{ADD <register> <value>} lägg till ett värde till ett register.
      \item \texttt{MUL <register> <value>} multiplicera ett register med ett värde och spara det i samma register.
    \end{itemize}

  \end{frame}

\end{document}
