\documentclass{beamer}
% vim: shiftwidth=2

\usecolortheme{seagull}
\usefonttheme{structurebold}

\usepackage[utf8]{inputenc}
\usepackage[swedish]{babel}

\usepackage{roboto}
\usepackage[scaled=0.9]{roboto-mono}
\usepackage[T1]{fontenc}

\usepackage[normalem]{ulem} % strikethrought
\usepackage{animate}
\usepackage{booktabs}
\usepackage{color}
\usepackage{fancyvrb}
\usepackage{icomma} % smart ',' in math mode
\usepackage{listings} % code listings
\usepackage{soul} % highlight and underline (etc)
\usepackage{url}
\usepackage{xcolor}

\lstset{basicstyle=\ttfamily\footnotesize,language=python,escapeinside=||}

\title{Seminarium 08}
\subtitle{Experimentering}
\date{8 november 2021}
\author{Gustav Sörnäs}

\setlength{\parskip}{1em}
\renewcommand{\baselinestretch}{1.2}

\AtBeginSection[] % Do nothing for \section*
{
\begin{frame}<beamer>
\frametitle{Planering}
\tableofcontents[currentsection]
\end{frame}
}

\begin{document}
  \frame{\titlepage}

  \begin{frame}{Dagens seminarium}

    \begin{itemize}
      \item Pseudokod
      \item Referenser
      \item Cellulära automater
    \end{itemize}

  \end{frame}

  \begin{frame}[fragile]{Pseudokod}

    Implementera algoritmen ``cocktail sort'' enligt pseudokoden i
    förberedelsematerialet på hemsidan.

    \texttt{https://seminarium.sörnäs.se}

  \end{frame}

  \begin{frame}[fragile]{Referenser}

    \begin{columns}
      \begin{column}{0.48\textwidth}
        \begin{lstlisting}
def add_two1(n):
    n + 2

def add_two2(n):
    n += 2

>>> a = 1
>>> add_two1(a)
>>> a
?  # 1
>>> add_two2(a)
>>> a
?  # 2
        \end{lstlisting}
      \end{column}
      \begin{column}{0.48\textwidth}
        \begin{lstlisting}    
def add_two3(lst):
    lst += [2]

def add_two4(lst):
    lst.append(2)

>>> b = [0, 1]
>>> add_two3(b)
>>> b
?  # 3
>>> add_two4(b)
>>> b
?  # 4
        \end{lstlisting}
      \end{column}
    \end{columns}

  \end{frame}

  \begin{frame}[fragile]{Referenser}

    \begin{lstlisting}
>>> a = []
>>> a.append(a)
>>> a
    \end{lstlisting}

  \end{frame}

  \begin{frame}{Cellulära automater}

    Vi har en en-dimensionell lista med ``celler'' som antingen lever eller är
    döda. För varje tidssteg kollar vi på cellen och dess två grannar för att se
    om den ska leva eller dö.

    \pause{}

    Exempel: En cell lever om och endast om dess vänstra granne levde förra
    tidssteget.

    \pause{}

    \small
    \texttt{00010000} \\
    \texttt{00001000} \\
    \texttt{00000100} \\
    \texttt{00000010}

    \pause{}

    \texttt{def rule(left, current, right):}\\\pause{}
    \hspace{4em}\texttt{return left}

  \end{frame}

  \begin{frame}[fragile]{Cellulära automater}

    Kopiera kodbasen från förberedelsematerialet och
    implementera \texttt{next\_state}.

    Fundera över vad som händer vid kanterna.

  \end{frame}

  \begin{frame}[fragile]{Cellulära automater}

    Skriv en predikatsfunktion som implementerar nedanstående sanningstabell.
    Vad för mönster bildas?

    \begin{tabular}{|c c c||c|}
      \hline
      left & current & right & next \\\hline
      0 & 0 & 0 & 0 \\
      0 & 0 & 1 & 1 \\
      0 & 1 & 0 & 1 \\
      0 & 1 & 1 & 1 \\\hline
      1 & 0 & 0 & 1 \\
      1 & 0 & 1 & 1 \\
      1 & 1 & 0 & 1 \\
      1 & 1 & 1 & 0 \\\hline
    \end{tabular}

  \end{frame}

  \begin{frame}{Cellulära automater i 2 dimensioner}

    Känt exempel: Conway's Game of Life.

    En cell

    \begin{enumerate}
      \item föds (död $\rightarrow$ levande) om den har 3 grannar vid liv
      \item överlever (levande $\rightarrow$ levande) om den har 2 eller 3 grannar vid liv
      \item dör eller fortsätter vara död annars.
    \end{enumerate}

  \end{frame}

\end{document}
