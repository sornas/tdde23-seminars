\documentclass{beamer}

\usecolortheme{seagull}
\usefonttheme{structurebold}

\usepackage[utf8]{inputenc}
\usepackage[swedish]{babel}

\usepackage{minted}  % code with syntax highlighting
\usepackage{url}

\title{Seminarium 02}
\subtitle{Experimentering}
\date{14 september 2021}

\setlength{\parskip}{1em}
\renewcommand{\baselinestretch}{1.2}

\AtBeginSection[] % Do nothing for \section*
{
\begin{frame}<beamer>
\frametitle{Planering}
\tableofcontents[currentsection]
\end{frame}
}

\begin{document}
  \frame{\titlepage}

  \section*{Introduktion}

  \begin{frame}
    \frametitle{Seminarieformen}

    Mer fokus på eget kodskrivande än ''Ett steg i taget'', men mer konkret än
    ''Diskussion och analys''.

    Fritt fram att byta spår under terminen.

    Innan seminariet: läs förberedelsematerialet och försök er på uppgifterna.

  \end{frame}

  \begin{frame}
    \frametitle{Dagens seminarium}

    \begin{itemize}
      \item Iteration.
      \item Uppgift: tupler och rekursion.
      \item Uppgift: palindrom.
      \item Uppgift: dictionaries.
      \item Diskussion.
    \end{itemize}

  \end{frame}

  \begin{frame}
    \frametitle{Iteration}

  \end{frame}

  \begin{frame}[fragile]
    \frametitle{\texttt{fibonacci}}

    \begin{columns}
      \begin{column}{0.48\textwidth}
        \begin{minted}[linenos]{python}
def f(x): return 2
        \end{minted}
      \end{column}%
      \begin{column}{0.48\textwidth}
        Frågor
      \end{column}%
    \end{columns}
  \end{frame}

\end{document}
