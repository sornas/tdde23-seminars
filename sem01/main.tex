\documentclass{beamer}

\usecolortheme{seagull}
\usefonttheme{structurebold}

\usepackage[utf8]{inputenc}
\usepackage[swedish]{babel}

\usepackage{minted}  % code with syntax highlighting

\title{Seminarium 01}
\subtitle{Satser och uttryck}
\date{7 september 2021}

\setlength{\parskip}{1em}
\renewcommand{\baselinestretch}{1.2}

\AtBeginSection[] % Do nothing for \section*
{
\begin{frame}<beamer>
\frametitle{Planering}
\tableofcontents[currentsection]
\end{frame}
}

\begin{document}
  \frame{\titlepage}

  \section*{Introduktion}

  \begin{frame}
    \frametitle{Lite om mig}

    Gustav Sörnäs \\
    \texttt{gusso230@student.liu.se}

    Ett år på Y, numera D3.

    Handledare för halva U1.b.

  \end{frame}

  \begin{frame}
    \frametitle{Lite om er}

    D1 och U1 (och kanske några äldre).

    Inga förväntade förkunskaper innan universitetet.

    Redo att testa er fram själva / i små grupper med egen dator.

  \end{frame}

  \begin{frame}
    \frametitle{Lite om seminarieformen}

    Mer fokus på eget kodskrivande än ''Ett steg i taget'', men mer konkret än
    ''Diskussion och analys''.

    Fritt fram att byta spår under terminen.

    Innan seminariet: läs förberedelsematerialet och försök er på uppgifterna.

  \end{frame}

  \begin{frame}
    \frametitle{Dagens seminarium}

    \begin{itemize}
      \item Satser och uttryck.
      \item Analysera \texttt{fibonacci}.
      \item Implementera en matematiskt definerad funktion.
      \item Uppgift: skriv en egen funktion.
      \item Variablers synlighet.
    \end{itemize}

    Inte så mycket kodskrivning. Diskussioner först i mindre grupper och sedan i
    helklass.

  \end{frame}

  \begin{frame}
    \frametitle{Satser och uttryck}

    Sats (\textit{statement}): En typ av \emph{instruktion} som kan utföras av
    Python. Enkla satser (tilldelning) och sammansatta satser (\texttt{for},
    \texttt{def}, \texttt{if}).

    Uttryck (\textit{expression}): Något som Python kan utvärdera till ett
    värde. Enkla uttryck (identifierare och \textit{literaler} (eller
    \textit{konstanter})) och sammansatta uttryck (additioner, subtraktioner,
    \dots).

  \end{frame}

  \begin{frame}[fragile]
    \frametitle{\texttt{fibonacci}}

    \begin{columns}
      \begin{column}{0.48\textwidth}
        \begin{minted}[linenos]{python}
def fibonacci(n):
    if n < 3:
        return n - 1
    a = 0
    b = 1

    for i in range(n - 2):
        c = b
        b = a + b
        a = c
    return b
        \end{minted}
      \end{column}%
      \begin{column}{0.48\textwidth}
        \begin{enumerate}
          \item Hur många satser finns i koden? Var finns dom?
          \item Hur många sammansatta uttryck finns i koden?
        \end{enumerate}
      \end{column}%
    \end{columns}
  \end{frame}

  \begin{frame}[fragile]
    \frametitle{\texttt{fibonacci}}

    \begin{columns}
      \begin{column}{0.48\textwidth}
        \begin{minted}[linenos]{python}
def fibonacci(n):
    if n < 3:
        return n - 1
    a = 0
    b = 1

    for i in range(n - 2):
        c = b
        b = a + b
        a = c
    return b
        \end{minted}
      \end{column}%
      \begin{column}{0.48\textwidth}
        \begin{enumerate}
          \item Vad gör funktionen?
          \item Vad returnerar funktionen?
          \item Vilken indata hanterar funktionen?
          \item Vad används variabeln \texttt{c} till? Vad skulle den kunna heta istället?
          \item Är returvärdet ett heltal eller flyttal?
          \item Vad kommer hända om ni matar in ett flyttal?
        \end{enumerate}
      \end{column}%
    \end{columns}
  \end{frame}

  \begin{frame}[fragile]
    \frametitle{\texttt{gcd}}

    \begin{columns}
      \begin{column}{0.48\textwidth}
        \begin{minted}{text}
+---
| gcd(a, 0) = a
| gcd(a, b) = gcd(b, a mod b)
+---
        \end{minted}
      \end{column}%
      \begin{column}{0.48\textwidth}
        \begin{enumerate}
          \item Beskriv hur beräkningen går till.
          \item Implementera \texttt{gcd} i Python.
          \item Räkna antalet satser och sammansatta uttryck i er Python-kod.
        \end{enumerate}
      \end{column}%
    \end{columns}
  \end{frame}

  \begin{frame}[fragile]
    \frametitle{\texttt{gcd}}

    \begin{columns}
      \begin{column}{0.48\textwidth}
        \begin{minted}[linenos]{python}
def gcd(a, b):
    if b == 0:
        return a
    return gcd(b, a % b)
        \end{minted}
      \end{column}%
      \begin{column}{0.48\textwidth}
        \begin{enumerate}
          \item Beskriv hur beräkningen går till.
          \item Implementera \texttt{gcd} i Python.
          \item Räkna antalet satser och sammansatta uttryck i er Python-kod.
        \end{enumerate}
      \end{column}%
    \end{columns}
  \end{frame}

  \begin{frame}
    \frametitle{Uppgift: \texttt{latest\_birthday\_before}}

    I grupper om minst fem personer, skriv en funktion som tar in två heltal
    (månad och dag) och returnerar namnet på den i gruppen vars födelsedag är
    närmast i tiden, räknat \emph{bakåt} från det inskickade datumet.

    \pause
    \vspace{1em}

    Exempel:

    \begin{columns}[T]
      \begin{column}{0.2\textwidth}
        \begin{enumerate}
          \item 20/8
          \item 15/8
          \item 18/7
        \end{enumerate}
      \end{column}%
      \begin{column}{0.78\textwidth}
        \texttt{latest\_birthday\_before(18, 8) -> (15, 8)}\\
        \texttt{latest\_birthday\_before(15, 8) -> (15, 8)}\\
        \texttt{latest\_birthday\_before(14, 8) -> (18, 7)}\\
        \pause
        \texttt{latest\_birthday\_before(10, 7) -> ?}
      \end{column}%
    \end{columns}
  \end{frame}

  \begin{frame}[fragile]
    \frametitle{Uppgift: }
    \begin{columns}
      \begin{column}{0.48\textwidth}
        \begin{minted}[fontsize=\scriptsize,linenos]{python}
g = 0
def outer(in1):
    a = "a"
    def inner(in2):
        print(a) # 1
        c = "c"
        a = "a from inner"
        print(in2) # 2
    print(c) # 3
    print(a) # 4
    inner("into inner")
    print(c) # 5
    print(a) # 6
    if g == 0:
        d = "g = 0"
    else:
        e = "g != 0"
    print(d) # 7
    print(e) # 8
        \end{minted}
      \end{column}%
      \begin{column}{0.48\textwidth}
        \begin{enumerate}
          \item Vilka print-satser kommer fungera? Vilka kommer att krascha?
          \item Vad skriver dom som fungerar ut?
        \end{enumerate}
      \end{column}%
    \end{columns}
  \end{frame}
\end{document}
