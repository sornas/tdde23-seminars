\documentclass{beamer}

% vim: shiftwidth=2

\usecolortheme{seagull}
\usefonttheme{structurebold}

\usepackage[utf8]{inputenc}
\usepackage[swedish]{babel}

\usepackage{roboto}
\usepackage[scaled=0.9]{roboto-mono}
\usepackage[T1]{fontenc}

\usepackage{booktabs}
\usepackage{color}
\usepackage{icomma} % smart ',' in math mode
\usepackage{soul} % highlight and underline (etc)
\usepackage[normalem]{ulem} % strikethrought
\usepackage{url}

\usepackage{listings} % code listings

\lstset{basicstyle=\ttfamily\footnotesize,language=python,escapeinside=||}

\title{Seminarium 05}
\subtitle{Experimentering}
\date{5 oktober 2021}
\author{Gustav Sörnäs}

\setlength{\parskip}{1em}
\renewcommand{\baselinestretch}{1.2}

\AtBeginSection[] % Do nothing for \section*
{
\begin{frame}<beamer>
\frametitle{Planering}
\tableofcontents[currentsection]
\end{frame}
}

\begin{document}
  \frame{\titlepage}

  \begin{frame}{Seminarieformen}

    Ibland läsa kod och diskutera frågor, ibland skriva kod i mindre grupper.
    Sedan diskussion i helklass.

    Innan seminariet: läs förberedelsematerialet och försök er på uppgifterna.

    Skicka in lösningar så vi kan diskutera i helklass:
    \texttt{seminarium.sörnäs.se}. Anonymt, sålänge du inte skriver ditt namn i
    koden :>

  \end{frame}

  \begin{frame}{Nästa seminarium}

    Repetitionstillfälle! Säg till om det är något ni vill repetera på mail
    till \url{gusso230@student.liu.se}.

  \end{frame}

  \begin{frame}{Dagens seminarium}

    \begin{itemize}
      \item Funktionell programmering
      \item Uppgift: högre ordningens funktioner
      \item Uppgift: lambdauttryck
      \item Uppgift: iteratorfunktioner
    \end{itemize}

  \end{frame}

  \begin{frame}{Funktionell programmering}

    Vad är funktionell programmering?

  \end{frame}

  \begin{frame}[fragile]{Funktionell programmering --- vanliga funktioner}

    \begin{lstlisting}
def plus_three(x):
    return isinstance(x, int) or isinstance(x, float)

>>> is_number(1)
True
>>> is_number(1.1)
True
>>> is_number("5")
False
>>> is_number([3.14])
False
    \end{lstlisting}

  \end{frame}

  \begin{frame}[fragile]{Funktionell programmering --- högre ordningens funktioner}

    \begin{lstlisting}
def is_number(x):
    return isinstance(x, int) or isinstance(x, float)

def test(first, pred):
    return pred(first)

>>> test(1, is_number)
True
>>> test(1.1, is_number)
True
>>> test("5", is_number)
False
>>> test([3.14], is_number)
False
    \end{lstlisting}

  \end{frame}

  \begin{frame}[fragile]{Funktionell programmering --- högre ordningens funktioner}

    \begin{lstlisting}
def is_number(x):
    return isinstance(x, int) or isinstance(x, float)

def both(first, second, pred):
    return pred(first) and pred(second)

>>> both(1, 2, is_number)
True
>>> both(1.1, 5, is_number)
True
>>> both("5", 10, is_number)
False
>>> both(1, [3.14], is_number)
False
    \end{lstlisting}

  \end{frame}

  \begin{frame}[fragile]{Funktionell programmering --- uppgift}

    Skriv en funktion \texttt{count(values, pred)} som går igenom en lista och
    räknar hur många av elementen som uppfyller ett predikat.

    Kom ihåg att ett predikat är en funktion som testar någonting och
    returnerar \texttt{True} eller \texttt{False}.

    \begin{lstlisting}
>>> count([1, 2, "3", 4, "5"], is_number)
3
>>> count([1, 2, [3]], is_number)
2
    \end{lstlisting}

  \end{frame}

  \begin{frame}[fragile]{Funktionell programmering --- rena funktioner}

    \begin{lstlisting}
def f(x):
    return 2 * x

>>> f(1)
2
>>> f(2)
4
f(1)
>>> 2
    \end{lstlisting}

  \end{frame}

  \begin{frame}[fragile]{Funktionell programmering --- sidoeffekter}

    \begin{lstlisting}
increment = 0
def g():
    increment += 1
    return increment

>>> g()
1
>>> g()
2
>>> g()
3
    \end{lstlisting}

  \end{frame}

  \begin{frame}[fragile]{Lambda-uttryck}

    Vi har sett att funktioner går att använda som värden.

    \begin{lstlisting}
def double(x):
    return x * 2

>>> f = double
>>> f(5)
10
>>> f
<function double at 0x7f0ff6cfbc10>
    \end{lstlisting}

  \end{frame}

  \begin{frame}[fragile]{Lambda-uttryck}

    \begin{lstlisting}
def double_f(x):
    return x * 2

double_l = lambda x: x * 2

>>> double_f(5)
10
>>> double_l(5)
10
    \end{lstlisting}

  \end{frame}

  \begin{frame}[fragile]{Lambda-uttryck}

    \begin{lstlisting}
def test(first, pred):
    return pred(first)

>>> test(1, lambda x: isinstance(x, int))
True
>>> test("2", lambda x: isinstance(x, int))
False
    \end{lstlisting}

  \end{frame}

  \begin{frame}[fragile]{Lambda-uttryck --- uppgift}

    Med hjälp av \texttt{count} från den tidigare uppgiften, skriv
    predikatfunktioner som lambdan för att räkna följande:

    \begin{enumerate}
      \item Antalet förekomster av strängen ''a''.
      \item Antalet listor som är två element långa.
      \item Antalet tal som är delbara med tre.
    \end{enumerate}

    \begin{lstlisting}
>>> count(["a", "B", "c", "a", "d"], ?)
2
>>> count([["a"], [1, 2], ["b", "c"]], ?)
2
>>> count([1, 2, 3, 4, 5, 6, 9], ?)
3
    \end{lstlisting}

    \begin{lstlisting}
lambda x: isinstance(x, int) or isinstance(x, float)
    \end{lstlisting}

  \end{frame}

  \begin{frame}{Iteratorfunktioner}

    Funktioner som hanterar iteratorer (t.ex. listor) på olika sätt.

    Vissa tar in funktioner som argument.

    Några exempel som kan vara bra att känna till:

    \begin{itemize}

      \item \texttt{list}: gör om en iterator till en lista. Väldigt vanlig.

      \item \texttt{map}: Applicera en funktion på varje element och ge tillbaka
      en iterator över retur-värdena.

    \end{itemize}

  \end{frame}

  \begin{frame}{Iteratorfunktioner --- andra bra grejer}

    Kolla upp vad dom gör genom att söka i Python-dokumentationen!

    \begin{itemize}
      \item \texttt{filter}
      \item \texttt{len}
      \item \texttt{min}
      \item \texttt{max}
      \item \texttt{enumerate}
    \end{itemize}

  \end{frame}

  \begin{frame}[fragile]{Iteratorfunktioner --- count}

    Skriv \texttt{count} från tidigare fast den här gången genom att kombinera
    inbyggda iteratorfunktioner!

    \begin{lstlisting}
def count(values, pred):
    return ?
    \end{lstlisting}

    \pause{}

    Tips: ni kommer använda 2 av följande 4 funktioner. Vilka? Och hur?

    \begin{itemize}
      \item \texttt{map}
      \item \texttt{filter}
      \item \texttt{len}
      \item \texttt{min}
    \end{itemize}

  \end{frame}

  %TODO mer uppgifter med iteratorfunktioner?

\end{document}
