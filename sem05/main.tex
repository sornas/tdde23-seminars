\documentclass{beamer}

% vim: shiftwidth=2

\usecolortheme{seagull}
\usefonttheme{structurebold}

\usepackage[utf8]{inputenc}
\usepackage[swedish]{babel}

\usepackage{roboto}
\usepackage[scaled=0.9]{roboto-mono}
\usepackage[T1]{fontenc}

\usepackage{booktabs}
\usepackage{color}
\usepackage{icomma} % smart ',' in math mode
\usepackage{soul} % highlight and underline (etc)
\usepackage[normalem]{ulem} % strikethrought
\usepackage{url}

\usepackage{listings} % code listings

\title{Seminarium 05}
\subtitle{Experimentering}
\date{5 oktober 2021}
\author{Gustav Sörnäs}

\setlength{\parskip}{1em}
\renewcommand{\baselinestretch}{1.2}

\AtBeginSection[] % Do nothing for \section*
{
\begin{frame}<beamer>
\frametitle{Planering}
\tableofcontents[currentsection]
\end{frame}
}

\begin{document}
  \frame{\titlepage}

  \begin{frame}{Seminarieformen}

    Ibland läsa kod och diskutera frågor, ibland skriva kod i mindre grupper.
    Sedan diskussion i helklass.

    Innan seminariet: läs förberedelsematerialet och försök er på uppgifterna.

    Skicka in lösningar så vi kan diskutera i helklass:
    \texttt{seminarium.sörnäs.se}. Anonymt, sålänge du inte skriver ditt namn i
    koden :>

  \end{frame}

  \begin{frame}{Nästa seminarium}

    Repetitionstillfälle! Säg till om det är något ni vill repetera, antingen
    nu, efter seminariet eller till mig på mail (\url{gusso230@student.liu.se}).

  \end{frame}

  \begin{frame}{Dagens seminarium}

    \begin{itemize}
      \item Uppgift: högre ordningens funktioner
      \item Uppgift: lambdauttryck
      \item Uppgift: inbyggda iteratorer
    \end{itemize}

  \end{frame}

\end{document}
