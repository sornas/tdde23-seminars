\documentclass{beamer}

\usecolortheme{seagull}
\usefonttheme{structurebold}

\usepackage[utf8]{inputenc}
\usepackage[swedish]{babel}

\usepackage{roboto}
\usepackage[scaled=0.9]{roboto-mono}
\usepackage[T1]{fontenc}

\usepackage{booktabs}
\usepackage{icomma} % smart ',' in math mode
\usepackage{minted} % code with syntax highlighting
\usepackage[normalem]{ulem} % strikethrought
\usepackage{url}

\title{Seminarium 03}
\subtitle{Experimentering}
\date{21 september 2021}
\author{Gustav Sörnäs}

\setlength{\parskip}{1em}
\renewcommand{\baselinestretch}{1.2}

\AtBeginSection[] % Do nothing for \section*
{
\begin{frame}<beamer>
\frametitle{Planering}
\tableofcontents[currentsection]
\end{frame}
}

\begin{document}
  \frame{\titlepage}

  \begin{frame}{Seminarieformen}

    Ibland läsa kod och diskutera frågor, ibland skriva kod i mindre grupper.
    Sedan diskussion i helklass.

    Innan seminariet: läs förberedelsematerialet och försök er på uppgifterna.

    Skicka in lösningar så vi kan diskutera i helklass:
    \texttt{seminarium.sörnäs.se}. Anonymt, sålänge du inte skriver ditt namn i
    koden :)

  \end{frame}

  \begin{frame}{Dagens seminarium}

    \begin{itemize}
      \item Läsbar kod
      \item Uppgift: metoder
      \item Uppgift: rekursiva listor
      \item Uppgift: dictionary
    \end{itemize}

  \end{frame}

  \begin{frame}{Läsbar kod}

    Korten i en kortlek är numrerade mellan 1 och 52. Skriv en loop som går
    igenom alla kort i kortleken.

  \end{frame}

  \begin{frame}[fragile]{Läsbar kod}

    \begin{minted}{python}
for i in range(1, 53):
    ...
    \end{minted}

  \end{frame}

  \begin{frame}[fragile]{Läsbar kod}

    \begin{minted}{python}
for card_nr in range(1, 53):
    ...
    \end{minted}

  \end{frame}

  \begin{frame}[fragile]{Läsbar kod}

    \begin{minted}{python}
deck_size = 52
for card_nr in range(1, deck_size + 1):
    ...
    \end{minted}

  \end{frame}

  \begin{frame}[fragile]{Läsbar kod}

    \begin{minted}{python}
deck_size = 52
for card_nr in range(1, deck_size + 1):
    ...

for card_nr in range(1, deck_size + 1):
    ...

for card_nr in range(1, deck_size + 1):
    ...
    \end{minted}

  \end{frame}

  \begin{frame}[fragile]{Läsbar kod}

    \begin{minted}{python}
DECK_SIZE = 52
for card_nr in range(1, DECK_SIZE + 1):
    ...

for card_nr in range(1, DECK_SIZE + 1):
    ...

for card_nr in range(1, DECK_SIZE + 1):
    ...
    \end{minted}

  \end{frame}

  \begin{frame}[fragile]{Metoder}

    Skriv funktioner som manipulerar strängar på olika sätt. Välj själva! \pause{}Några
    förslag:

    {
      \ttfamily
      \small
      \begin{tabular}{l l}
        Indata & Utdata \\\toprule
        Hello noob & h3ll0 n00b \\
        Super sale & !!!SUPER SALE!!! \\
        snake\_to\_camel & snakeToCamel \\
        /linux/to/windows & C:\textbackslash{}linux\textbackslash{}to\textbackslash{}windows \\
        Por que no los dos? & ¿Por que no los dos? \\
        Jag är vilse & Jojagog äror vovilolsose \\
        Best idea ever & Best. Idea. Ever. \\
      \end{tabular}
    }

    \pause{}

    \begin{minted}{python}
>>> leetspeak("Hello noob")
"h3ll0 n00b"
    \end{minted}

    \vspace{-1em}

    \texttt{seminarium.sörnäs.se}

  \end{frame}

  \begin{frame}[fragile]{Rekursiva listor}

    Iteration för att gå igenom en lista:

    \begin{minted}{python}
def print_values(values):
    for value in values:
        print(value)

>>> numbers = [1, 2, 5]
    \end{minted}

  \end{frame}

  \begin{frame}[fragile]{Rekursiva listor}

    Rekursion för att gå igenom en lista:

    \begin{minted}{python}
def print_values(values):
    if not values:
        return
    else:
        print(values[0])
        print_values(values[1:])

>>> numbers = [1, 2, 5]
    \end{minted}

  \end{frame}

  \begin{frame}[fragile]{Rekursiva listor}

    Ibland har vi t.ex. listor i listor, en typ av \emph{rekursiv struktur}.

    \pause{}

    for-loop för att gå igenom den yttre listan och rekursion för att gå igenom de inre listorna:

    \pause{}

    \begin{minted}{python}
def print_values(values):
    for value in values:
        if isinstance(value, list):
            print_values(values)
        else:
            print(value)

>>> numbers = [1, [2], [[[5]], 6]]

    \end{minted}

  \end{frame}

  \begin{frame}[fragile]{Rekursiva listor}

    Rekursion för att gå igenom både den yttre listan och de inre listorna:

    \begin{minted}{python}
def print_values(values):
    if not values:
        return
    elif isinstance(values[0], list):
        print_values(values[0])
    else:
        print(values[0])
    print_values(values[1:])


>>> numbers = [1, [2], [[[5]], 6]]

    \end{minted}

  \end{frame}

  %TODO prata om när den ena föredras framför den andra?

  \begin{frame}[fragile]{Rekursiva listor --- uppgift}

    Skriv en funktion \texttt{find\_length(length, values)} som letar efter ord
      av längd \texttt{length} i en lista (\texttt{values}) och dess underlistor
      (eventuellt med ytterligare listor).

    Tips:

    \vspace{-1em}

    \begin{itemize}

      \item Rekursionen kommer alltid bara kolla på första elementet
      (\texttt{values[0]}) och resten av listan (\texttt{values[1:]}).

      \item Fundera på vad som händer om:

      \begin{enumerate}
        \item Den inskickade listan är tom.
        \item Det första elementet är en lista.
        \item Det första elementet inte är en lista. \\
        \dots{}
      \end{enumerate}

      Kan vissa av de här ske samtidigt? Vad händer då?

    \end{itemize}

  \end{frame}

  \begin{frame}[fragile]{Dictionary --- uppgift}

    Skriv en funktion \texttt{flatten\_dict} som givet ett dictionary av
    dictionaries returnerar ett nytt ''platt'' dictionary.

    \begin{columns}
      \begin{column}{0.48\textwidth}
        \begin{minted}[fontsize=\footnotesize]{python}
>>> data = {
        "a": {
            "a": {
                "python": 1,
                "java": 2
            },
            "b": {}
        },
        "b": {
            "c": 3
        },
    }
>>> flatten_dict(data)
{'python': 1, 'java': 2, 'c': 3}
        \end{minted}
      \end{column}
      \begin{column}{0.48\textwidth}

        \begin{itemize}
          \item \texttt{a | b} kombinerar två dictionaries. Testa!
        \end{itemize}

      \end{column}
    \end{columns}

  \end{frame}

\end{document}
