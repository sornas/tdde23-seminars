\documentclass{beamer}
% vim: shiftwidth=2

\usecolortheme{seagull}
\usefonttheme{structurebold}

\usepackage[utf8]{inputenc}
\usepackage[swedish]{babel}

\usepackage{roboto}
\usepackage[scaled=0.9]{roboto-mono}
\usepackage[T1]{fontenc}

\usepackage{booktabs}
\usepackage{color}
\usepackage{icomma} % smart ',' in math mode
\usepackage{soul} % highlight and underline (etc)
\usepackage[normalem]{ulem} % strikethrought
\usepackage{url}

\usepackage{listings} % code listings

\lstset{basicstyle=\ttfamily\footnotesize,language=python,escapeinside=||}

\title{Seminarium 06}
\subtitle{Experimentering}
\date{5 oktober 2021}
\author{Gustav Sörnäs}

\setlength{\parskip}{1em}
\renewcommand{\baselinestretch}{1.2}

\AtBeginSection[] % Do nothing for \section*
{
\begin{frame}<beamer>
\frametitle{Planering}
\tableofcontents[currentsection]
\end{frame}
}

\begin{document}
  \frame{\titlepage}

  \begin{frame}{Seminarieformen}

    Innan seminariet: läs förberedelsematerialet och försök er på uppgifterna.

    Skicka in lösningar så vi kan diskutera i helklass:
    \texttt{seminarium.sörnäs.se}. Anonymt, sålänge du inte skriver ditt namn i
    koden :>

  \end{frame}

  \begin{frame}{Dagens seminarium}

    \begin{itemize}
      \item Uppgift: testning
      \item Uppgift: gammal tentauppgift
    \end{itemize}

  \end{frame}

  \begin{frame}{Testning}

    \begin{itemize}
      \item Vad är ett fel?
      \item När och hur upptäcks fel?
      \item Vad kan orsaka fel?
      \item Hur man kan lokalisera och åtgärda fel?
    \end{itemize}

  \end{frame}

  \begin{frame}[fragile]{Testing --- uppgift}

    \begin{lstlisting}
def count(seq):
    """
    Counts the number of elements in a given list
    including elements in inner lists
    """
    if not seq:
        return 0
    elif isinstance(seq[0], list):
        return count(seq[0]) + count(seq[1:])
    else:
        return 1 + count(seq[1:])
    \end{lstlisting}

    Del 1: Skriv en funktion \texttt{test\_count} (utan parametrar) som testar
    \texttt{count} med olika indata. Använd \emph{inte} \texttt{assert} utan se till
    att funktionen returnerar något som går att använda.

    % Fundera över hur olika testfall ska sparas, alltså vilka datastrukturer.

  \end{frame}

  \begin{frame}[fragile]{Uppgift: delsekvenser}

    Skriv en funktion \texttt{subseq(seq)} som tar in en sekvens (endast tupler)
    och returnerar \textit{mängden} av alla delsekvenser. Exempel:

    \pause{}
    \begin{lstlisting}
assert subsequences((1)) == {(1,), ()}
    \end{lstlisting}
    \pause{}
    \begin{lstlisting}
assert subsequences((1, 2)) == {(1,), (1, 2), (2,), ()}
    \end{lstlisting}
    \pause{}
    \begin{lstlisting}
assert subsequences((1, 1)) == {(1,), (1, 1), ()}
    \end{lstlisting}
    \pause{}
    \begin{lstlisting}
assert subsequences((1, 2, 3)) == {(1, 3), (1, 2),
                                   (2,), (1, 2, 3),
                                   (2, 3), (1,),
                                   (), (3,)}
    \end{lstlisting}

    \pause{}
    Rekursion eller iteration?

  \end{frame}

\end{document}
