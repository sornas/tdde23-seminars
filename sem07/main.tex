\documentclass{beamer}
% vim: shiftwidth=2

\usecolortheme{seagull}
\usefonttheme{structurebold}

\usepackage[utf8]{inputenc}
\usepackage[swedish]{babel}

\usepackage{roboto}
\usepackage[scaled=0.9]{roboto-mono}
\usepackage[T1]{fontenc}

\usepackage[normalem]{ulem} % strikethrought
\usepackage{booktabs}
\usepackage{color}
\usepackage{fancyvrb}
\usepackage{icomma} % smart ',' in math mode
\usepackage{listings} % code listings
\usepackage{soul} % highlight and underline (etc)
\usepackage{url}
\usepackage{xcolor}

\lstset{basicstyle=\ttfamily\footnotesize,language=python,escapeinside=||}

\title{Seminarium 07}
\subtitle{Experimentering}
\date{2 november 2021}
\author{Gustav Sörnäs}

\setlength{\parskip}{1em}
\renewcommand{\baselinestretch}{1.2}

\AtBeginSection[] % Do nothing for \section*
{
\begin{frame}<beamer>
\frametitle{Planering}
\tableofcontents[currentsection]
\end{frame}
}

\begin{document}
  \frame{\titlepage}

  \begin{frame}{Dagens seminarium}

    \begin{itemize}
      \item Uppgift: EBNF
      \item Uppgift: Identifiera ord
      \item Uppgift: Prefix
      \item Uppgift: Primtal i fibonacci
      \item Uppgift: Relativ sökväg
    \end{itemize}

  \end{frame}

  \begin{frame}[fragile]{EBNF}
    \begin{itemize}
      \item Terminal: något som bara innehåller tecken.
      \item Regel: något som innehåller regler och terminaler.
    \end{itemize}

    \begin{Verbatim}[commandchars=\\\{\}]
A = 'a' | 'b' | '' ;
S = 'a', A | 'b', A ;
    \end{Verbatim}
  \end{frame}

  \begin{frame}[fragile]{EBNF}
    \begin{itemize}
      \item \colorbox{orange!80}{Terminal}: något som bara innehåller tecken.
      \item \colorbox{blue!50}{Regel}: något som innehåller regler och terminaler.
    \end{itemize}

    \begin{Verbatim}[commandchars=\\\{\}]
\colorbox{orange!80}{A} = \colorbox{orange!80}{"a"} | \colorbox{orange!80}{"b"} | \colorbox{orange!80}{""} ;
\colorbox{blue!50}{S} = \colorbox{orange!80}{"a"}, \colorbox{blue!50}{A} | \colorbox{orange!80}{"b"}, \colorbox{blue!50}{A} ;
    \end{Verbatim}
  \end{frame}

  \begin{frame}[fragile]{EBNF}
    \begin{Verbatim}[commandchars=\\\{\}]
A = 'a' | 'b' | '' ;
S = 'a', A | 'b', A ;
    \end{Verbatim}

    \texttt{S} är startsymbolen, så alla ord som \emph{uppfyller} \texttt{S} är
    en del av språket.

    \begin{itemize}
      \item \texttt{ba} är i språket.
      \item \texttt{baa} är inte i språket.
    \end{itemize}

  \end{frame}

  \begin{frame}[fragile]{EBNF}
    \begin{Verbatim}[commandchars=\\\{\}]
A = \colorbox{blue!50}{'a'} | 'b' | '' ;
S = 'a', A | \colorbox{blue!50}{'b', A} ;
    \end{Verbatim}

    \texttt{S} är startsymbolen, så alla ord som \emph{uppfyller} \texttt{S} är
    en del av språket.

    \begin{itemize}
      \item \texttt{aa} är i språket.
      \item \texttt{aab} är inte i språket.
    \end{itemize}

  \end{frame}

  \begin{frame}[fragile]{EBNF}
    \begin{Verbatim}[commandchars=\\\{\}]
A = 'a' | 'b' | '' ;
S = 'a', A | 'b', A ;
    \end{Verbatim}

    Alla ord i språket: \texttt{a, aa, ab, b, ba, bb}.
  \end{frame}

  \begin{frame}[fragile]{EBNF -- Uppgift}
    Ange alla ord som ingår i följande språk:

    \begin{enumerate}
      \item \begin{Verbatim}
S = 'a', A | 'b', B ;
A = 'a', B | '' ;
B = 'a' | 'b' ;
      \end{Verbatim}
      \item \begin{Verbatim}
S = 'a', S, 'a' | 'b', S, 'b' | 'a' | 'b' | '' ;
      \end{Verbatim}
    \end{enumerate}
  \end{frame}

  \begin{frame}[fragile]{Identifiera ord}

    \begin{enumerate}
      \item \begin{Verbatim}
S = 'a', A | 'b', B ;
A = 'a', B | '' ;
B = 'a' | 'b' ;
      \end{Verbatim}
      \item \begin{Verbatim}
S = 'a', S, 'a' | 'b', S, 'b' | 'a' | 'b' | '' ;
      \end{Verbatim}
    \end{enumerate}

    Skriv kod som identifierar om en sträng ingår i språk 1 respektive språk 2.

    Tips: Det är svårt att lösa den här uppgiften helt generellt. Fokusera på språk 1 och 2.

  \end{frame}

  \begin{frame}[fragile]{Prefix}

    Skriv funktionen \texttt{contains\_prefixes} som tar två listor med strängar
    och returnerar om \emph{varje} sträng i den första listan är ett prefix till
    \emph{någon} av strängarna i den andra listan.

    \begin{lstlisting}
>>> contains_prefixes(["hej"], ["hejsan", "asdfasdf"])
True
>>> contains_prefixes(["hej", "sdf"], ["hejsan", "asdfasdf"])
False
    \end{lstlisting}

  \end{frame}

  \begin{frame}{Primtal i fibonacci-sekvensen}

    Skriv en funktion som returnerar en lista med tal ur fibonacci-serien som
    dessutom är primtal.

    Funktionen behöver ta in en heltalsparameter men ni får själva bestämma vad
    den betyder. Det viktiga är att den på något sätt begränsar storleken på
    listan.

    Dom tio första primtalen i fibonacci-serien:
    \texttt{[3, 5, 13, 89, 233, 1597, 28657, 514229, 433494437, 2971215073]}

  \end{frame}

  \begin{frame}[fragile]{Relativ sökväg}

    Skriv en funktion som givet två sökvägar returnerar en relativ sökväg från
    den första till den andra.

    \begin{lstlisting}
/home/gustav/liu/tdde23/seminarium
/home/gustav/dev/impa
=>
../../../dev/impa
    \end{lstlisting}

  \end{frame}

\end{document}
